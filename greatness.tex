\documentclass[ebook,12pt,oneside,openany]{memoir}
\usepackage[utf8x]{inputenc}
\usepackage[english]{babel}
\usepackage{url}

\begin{document}

\title{Why Greatness Cannot Be Planned}
\maketitle

\section{Questioning Objectives}
Whether you're an executive, a scientist, a student, or even just a single looking for a date, once your objective is defined usually the next step is to put all your energy into achieving it. 
The purpose of all the measuring is to help us figure out whether we're heading in the right direction.
The process of setting an objective, attempting to achieve it, and measuring progress along the way has become the primary route to achievement in our culture.

One of the reasons that objectives aren't often questioned is that they work perfectly well for more modest pursuits. But as objectives become more ambitious, reaching them becomes less promising-and that's where the argument becomes more interesting.

Why do we have to wander? Why can't we just go directly to the location of the most powerful machine of all? 
Stepping stones are portals to the next level of possibility. Before we get there, we have to find the stepping stones.

Objectives are well and good when they are sufficiently modest, but things get a lot more complicated when they're more ambitious.
In fact, objectives actually become obstacles towards more exciting achievements, like those involving discovery, creativity, invention or innovation-or even achieving true happiness.
In other words, the greatest achievements become less likely when they are made objectives.

Sometimes the best way to achieve something great is to stop trying to achieve a particular great thing.
In other words, greatness is possible if you are willing to stop demanding what that greatness should be.

\end{document}