\documentclass[ebook,12pt,oneside,openany]{memoir}
\usepackage[utf8x]{inputenc}
\usepackage[english]{babel}
\usepackage{url}

\begin{document}

\title{The Fearless Organization}
\maketitle

\section{Why Fear Is Not an Effective Motivator}
Many managers-both consciously and not-still believe in the power of fear to motivate.
They assume that people who are afraid (of management or of the consequences of underperforming) will work hard to avoid unpleasant consequences, and good things will happen.
This might make sense if the work is straightforward and the worker is unlikely to run into any problems or have any ideas for improvement.
But for jobs where learning or collaboration is required for success, fear is not an effective motivator.

Psychological safety describes a belief that neither the formal nor informal consequences of interpersonal risks, like asking for help or admitting a failure, will be punitive.
In psychologically safe environments, people believe that if they make a mistake or ask for help, others will not react badly. Instead, candor is both allowed and expected.
Psychological safety exists when people feel their workplace is an environment where they can speak up, offer ideas, and ask questions without fear of being punished or embarrassed. 
Is this a place where new ideas are welcomed and built upon? Or picked apart and ridiculed? Will your colleagues embarrass or punish you for offering a different point of view?
Will they think less of you for admitting you don't understand something?

\end{document}