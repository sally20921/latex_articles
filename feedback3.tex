\documentclass[ebook,12pt,oneside,openany]{memoir}
\usepackage[utf8x]{inputenc}
\usepackage[english]{babel}
\usepackage{url}

\begin{document}

\title{No More Feedback}
\maketitle

\section{A Technology Of Change}
\subsection{The Beginning of Change}
I know without asking that they have embraced one or more of the hundred toxic practices in existence, nearly one-third of which I briefly described in my book, \textit{The Regenerative Business}.
Of these practice, feedback is one of the most counterproductice and widespread.

``What enables people to contribute to their full potential?'' An honest effort to answer will follow the same thread and arrive at the same basic conclusion about human development:
Success of any kind arises from the ability we all share to develop \textit{three core capabilities}, and our willingness to design work systems and practices that support their development and foster them in our culture.

The technology of change I am offering here is a method for thinking and developing as a human with fully realized core capabilities.
This is the only practical route to developing facility with discernment and change, and it is a far better route than feedback for nurturing conscious employees.

\subsection{Three Core Human Capabilities}
All people have far more potential than they achieve in their short lives. We do not know the foundational capacities that, if fully developed, would give people the extraordinary ability to grow themselves.
Developing these capacities also would engender more courage and vision.

For the most part, we work on the wrong things. Because we do not work directly on the underlying capacities necessary to achieve worthwhile aims, we undermine their realization.
This is what I mean by toxic practices. These ways of working are lethal to the foundational capacities that make us fully ourselves and provide springboards for great lives, allowing us to achieve our full potential and make beneficial contributions to others.

The three human capacities are \textit{locus of control, scope of considering, and source of agency}.
They are innate in all people but most of our societal roles offer few opportunities to develop them.
It limits our life experiences and our ability to advance into ever more responsible roles in our families, organizations, and communities.

\textbf{Locus of control} speaks to the degree to which we experience and exercise control over our own lives, particularly on the direction of our self-development and our resilience to adversity.
\textbf{Scope of considering} relates to what we take into account in our actions and endeavors, especially in relation to other people and living beings.
We may be self-centerd and inwardly focused, or we may consider the effects of our actions on other individuals and groups or entire living systems.
The difference is between self-centered focus on oneself alone and a systems-actualizing focus on evolving a larger whole-marriage, family, organization, community, industry, ecosystem, planet-in order to create beneficial changes.
\textbf{Source of agency} refers to where we find authority for our initiative or actions. We may rely almost exclusively on the authority of others to direct us or we may have within us the will to initiate action ourselves and follow through with self-directed efforts.
The more we are able to direct ourselves, the better our capability to connect to larger systems and help actualize them.

Locus of control moves from external, seeing our lives are determined by others, to internal, taking accountability for what we exercise in terms of outcomes and level of direction.
We are able to go back and forth between external and internal but we have usually settled into a tendency toward one or the other by the time we reach adulthood.

Scope of considering is in a sense the opportunity to get perspective on the internal and external events of our lives.
When we consider only ourselves, every situation we encounter is all about us. On the other hand, if we are sensitive to otehrs in our world and to other forms of life, we have developed a degree of external considering.

Source of agency is likewise very fluid but tends to be directed by beliefs we hold about our roles in the world and who has power or influence us.
When we live according to an authoritarian worldview, we often wait for important others to activate, or direct, or stop us. 
But as we become driven internally and come to believe that the world is ours, we begin to move toward a life devoted to stepping up and making a difference. We develop personal agency, the courage to demand more of ourselves and respond to internal calls that connect us to powerful opportunities.

Without conscious development, these three core capacities may stay nascent our entire lives, diminishing us and limiting the contribution we can make. 
But if we are willing to develop them by ourselves or hand-in-hand with organizations or communities designed to work on such development, we may be astounded by how much we can grow and how fully ourselves we can become.
The challenge is to avoid the practices and systems that steer us toward a smaller perspectice and set of pursuits.

Internal locus of control is the certainty that responsibility for any and all outcomes rests with oneself. A person cannot control everything that happens around or within them, but they can take responsibility for their reactions and self-development in the midst of it all.
For a person with well-developed internal locus of control, losing a job or ending a marriage becomes an opportunity to rethink life. Research shows that people with this capacity are far happier, even in the face of life's most unnerving or threatening challenges.

External locus of control, presents itself as passivity and victimhood. People with external locus of control do not step up. 

External considering is exercising our connection to others, not being concerned about only ourselves. Those who are not practicing external considering may give others the sensation that they are unseen and unheard, pawns in a bureaucratic game, treated as if they were not fully alive.

Personal agency is the effective exercise of the will to act and make changes that benefit ourselves and others. It is the opposite of the passivity that executives complain so much about in their workers.
When fully developed, personal agency transforms individuals into leaders, go-getters, and change agents. People acting from personal agency cannot tolerate sitting passively by when what is failing is not confronted or put right.
The opposite of this kind of person is one who will work usually only whenfear dictates exactly what must be done.
The difference between the two is easily seen as the difference between willingness to take risks and avoid them.

The important question in any moment is, ``Which way am I moving?''

\subsection{Looking Ahead-Alternatives to Feedback and Other Toxic Practices}
The alternatives to toxic practices and systems are all based on the ability to engage in self-reflection, which becomes more accurate, motivational and innovative when development of the three capacities is remembered as an ultimate aim.
What if people could see themselves and their behaviors so clearly that no feedback was needed to guide them in their work? When self-observation and reflection are consistently tied to what I call ``Big Promises'' to stakeholders, better, more compelling motives are the natural result.
When we ask ourselves about the difference we are making in other people's lives, we cannot help but be moved to think more creatively, to do more and do it better.
It is a matter of asking questions like, ``Where do I need to be more ambitious and innovative for them?''

What if we primarily taught people to engage in self-reflection, self-direction, and self-management of their own growth and development? What if we asked them to play more powerful roles and to be part of growing one another's capacities in work that required no outside feedback or input?

The drive to be self-regulating, self-improving, and self-challenging is innate in all of us and ready to be tapped.
Put a person in charge of their own assessment, and they will soar.
I found that I could awaken their core capacities and bring them to life in a very short time by switching to work designs based on self-regulation, self-direction, and self-motivation in the form of Big Promises.

Put people in charge of their own change in collaboration with others. Ask good questions (with no attachment to correct answers) to open the way for them to design their own individual and team evolution, always using the educational process as the springboard for innovation.

\begin{itemize}
    \item What are our working premises and principles and how do we validate them?
    \item How do we develop the capacity in people to work developmentally?
    \item How do we build a culture to support this way of working, one in which reflection and assessment comes from ourselves and not from others?
    \item How do we purge all of our assessment processes of the biases and attachments that make feedback so toxic?
\end{itemize}

\section{Feedback}
Feedbakc is offering or receiving opinions, impressions and assessments of attitudes and behavior from others in any form or context.
Feedback is based on the idea that: 1) people cannot see themselves clearly and cannot objectively assess the effects of their actions, and 2) external observers-assumed to be clear-eyed, unprejudiced, and reliably objective-must do this work for them.
We are ready to look at the history of feedback with three objectives in mind:
\begin{enumerate}
    \item To see why resourcing development of the three core capacities may be the most important and generous gift one can give to another person
    \item To understand that feedback and other limiting practices are the surest way to discourage the development of capacities that make these promises possible
\end{enumerate}
I hope to make it evident that feedback not only slows down this development, it may also completely derail it. My experience and research show that these effects occur no matter how well feedback is practiced,
regardles of training and preparation or good intentions. 
\textbf{Input from other people tends to trigger our need to belong, a response demanded by our brain for survival. This causes us to give external input more weight than our own reflections, which encourages us to work toward others' ideas and suggestions instead of toward our own. Over time, we become dependent on input from outside ourselves.}
It becomes our default for confirmation and motivation, and it can happen to anyone. 

More importantly, feedback undermines and erodes our capacity for self-reflection and self-direction. 
This is true even of people who loudly proclaim the good feedback does them by providing information they never would have uncovered on their own. This kidn of declaration is often motivated by a push from the brain to establish belonging.
It is often heartfelt and sincere but not nearly as powerful as the thrill of self-discovery.

By ``really bad stuff'', I mean encounters with others who tell you exactly what they think of you, evaluating you based only on their privately held thoughts, which can seem terribly biased and strand you with no positive way forward.
The prevailing idea that it is always better to get the real truth about yourself from someone lese, someone with an objective eye, has limited people's ability to independently develop their sense of self.
People are discouraged before they even begin the destabilizing yet thoroughly rewarding effort of finding their truth for themselves.
Every one of us can learn to observe ourselves, reflect on what we see, and develop deep insights into our thinking, feelings, and behaviors. we can do a much better job of that than others can do for us. 
External input tends to shut down the growth and exercise of personal agency, respect and appreciation for the contributions of thers, and belief in our responsibility for what goes on around us or happens to us.
The benefits of feedback are purely mythological.

It's not a question of whether feedback works. It does! The need to belong will drive changes in behavior whether they are in line with people's intentions or not. The real questions are: Does feedback encourage and make it possible for all people to develop the three core human capacities-locus of control, scope of considering, and source of agency-which drives our ableness to live fully in the world and take action?
It is of utmost importance that its leaders ask themselves some tough questions. For example:
\begin{itemize}
    \item Do people seem distracted or anxious in the periods leading up to feedback sessions, either because they will receive it? 
    \item Have I ever suspected that an unexamined judgement has coalesced around an individual's behavior and is showing up inappropriately in feedback?
    \item Do our feedback processes often leave people feeling isolated?
\end{itemize}

\subsection{Feedback and Human Capacity}
Although feedback systems have existed since antiquity, it was not until the Industrial Revolution of the eighteenth and nineteenth centuries that the notion ``to feed back'' was recognized as a universal abstraction or concept.
To develop the three human capacities, a person must be self-directed. Inner processing is the only way to shift from external to internal locus of control, to broaden one's scope of considering, and to build personal agency.
Only I can examine, interpret, understand and move forward on what I experience.
No one can do that for me. Only when people come to see that they are giving away their control to others can they begin to break the cycle.
Feedback, by definition, is other-directed; we tend toward increased external locus of control and internal considering when we are constantly fed other people's interpretations of our ideas, emotional expression, and behavior.
Feedback also makes us wonder and worry about how others see and value us, which displaces our concern for others, and causess us to become more and more self-absored and self-centered.

The most fundamental difference between a machine and a human being is that a machine is a closed system and a human is an open system. Human beings work reciprocally with their environments and maintain relationships with them. 
In the eyes of behaviorism, people simply could not be self-observing, self-understanding and self-directed by drawing guidance within themselves.
Humans do not have the same clear boundaries with their environments as machines. It is not always clear who controls what. Humans engage and interprete their environments with intellect and emotion and observe their own proceses as they simultaneously reflect and take action.

\section{Downsides to Feedback}
\subsection{Flaws in the Theory of Objective Feedback}
The latest research tells us that we are bad at understanding others because we have conditional biases-and also because we tend to project our own shortfalls onto others.
This tendency to be blind to our own nature and project it onto others is well documented and widely accepted by psychologists, as is the fact that those on the receiving end of projections usually fail to notice them.
As a result of some cognitive biases-the recipients may accept false information and observations about themselves that do not fit their situations or behaviors.
That is truly deadly to the clarity and objectivity of feedback in performance appraisals, coaching, or any process in which one person is entrusted with making and sharking objective observations about another person.

For some reaso, we assume that the observer is neutral and can see the truth while the person under observation is clouded by their own biases and interpretation.

\subsection{The Limits of Objectivity: Cognitive Biases}
You might assume that if multiple people provide the same input, the output will be accurate. But group-projection processes exist that work strongly against objectivity. In fact, several cognitive biases can sway people collectively toward false conclusions.
When we make judgements and decisions, we like to think that we are objective, logical and capable of taking and evaluating all of the available information. Unfortunately, the less self-aware we are, the more likely we are to be tripped up by biases, which lead us to make poor decisions and bad judgements.
\begin{itemize}
    \item Confirmation Bias: We tend to believe that we know people and things well enough to discern significant differences in their beahaviors over time. In fact, we form and hold ideas early on and usually fail to notice or question changes, especially when what we think we know matches our strongly held views.
    This bias favors information that conforms to our existing beliefs and discounts evidence that does not conform.
    \item Availability Heuristic: What we are actually valuing is not accuracy but looking smart by coming up with quick answers. This places greater value on speed and quantity than on quality. We give greater credence to information that comes to us quickly than to what occurs to us later, and we tend to overestimate the probability that what we observed in the moment will reoccur.
    We project our current ideas into the future, and this makes us even less likely to see significant changes.
    \item Halo Effect: We do tend to form an impression when we meet someone that changes very little over time. In fact, we tend to mistake an immediate, overall impression for a reliable assessment of a person.
    This impression then influences how we feel and think about his or her character going forward. 
    \item Self-Serving Bias: This is the tendency to blame external forces when bad things happen and to credit ourselves when good things happen. We tend to gauge our own chances of being benefitted or harmed by the effects of our feedback rather than its usefulness to the recipient.
    \item Attentional Bias: This is the tendency to pay attention to some things while simultaneously ignoring others. We may pay attention to whether and how often they agree with us while also ignoring their ideas, especially when we are envious of them. We tend to downplay or ignore what is uncomfortable when we are interpreting observations and experience.
    \item Functional Fixedness: We often develop a tendency to believe that a familiar object can work only in the particular way we have seen it work in the past. Seeing people as fixed in their skills and characters, especially if those assessments are dated, casues us to judge peopel based on old ideas versus seeing their potential. This limits our ability to support the growth of others with feedback.
    \item Anchoring Bias: Anchoring is the tendency to rely too heavily on the very first experience or information that comes one's way, rather than waiting to learn more before forming an opinion. We view the way we do something for the first time, as either the best or worst way to do the same thing in the future. We can fail to see that how someone else does the thing may be more effective than our way. Our minds are closed to further learning, which may rob our feedback value.
    \item Misinformation Effect: Misinformation received after an event or an experience with a person can interfere with your original memory of the event or person. It is easy to be swayed by second-hand information, and this can lead to the development of conformity in group thinking in preparation for a feedback cycle.
\end{itemize}

Two biases that cross all other known biases and are invisible to those of us who have not experienced them are race and gender. Groups tend to unconsciously collude around these biases, making it seem that there is consensus in feedback that incorporates them.
Undeserved suspicious attention paid to people of color who are not in any way behaving in ways that need to be monitored or managed is the result of these biases embedded in whole swathes of populations. There is no reason to believe organizations have succeeded in excluding the effects of these biases from their feedback programs.
Bias marches through every feedback structure and process. 

Because those people whose observing and thinking are most colored by cognitive, race, and gender biases have little capacity to notice them, they are likely to go undetected in feedback processes. Groups of reviewers are far more likely to fall into group think than are to get to a truth.

\subsection{Feedback and Human Self-Regulation}
What happens when you manage people and business with feedback as if they were closed systems? It enhances the false premise that people cannot see into themselves and understand their own behavior-a false notion that has been drilled into them for most of their lives.
And it causes a painful disconnect between how people experience themselves and what they hear from others.

People were becoming less connected to business outcomes and more concerned about themselves and how others saw them. This is self-centered internal considering, the opposite of system-actualizing external considering. 

Many employees' attention seemed to be increasingly focused on fitting in. The people's desire to look like others-rather than express their uniqueness was more and more prominent and becoming a competitive way of viewing performance.

A developmental approach is based on a paradigm that sees every living entity as having unique essence of being that is searching for channels and means of expression. This way of seeing the whole can easily become lost if there is a shift back to the internal focus on competitive ventures.

\section{Premises For Designing Developmental Work Systems}

\end{document}