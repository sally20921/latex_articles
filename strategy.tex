\documentclass[ebook,12pt,oneside,openany]{memoir}
\usepackage[utf8x]{inputenc}
\usepackage[english]{babel}
\usepackage{url}

\begin{document}

\title{The Strategy Paradox}
\maketitle

Here is a puzzling fact: the best-performing firms often have more in common with humiliated bankrupts than with companies that have managed merely to survive.
In fact, the very traits we have come to identify as determinants of high achievement are also the ingredients of total collapse. 
And so it turns out that, behaviorally at least, the opposite of success is not failure, but mediocrity.
Therein lies the strategy paradox: the same behaviors and characteristics that maximize a firm's probability of notable success also maximize its probability of total failure.
\textbf{The Similarity of Opposites} Many opposites are not nearly as different as they first appear. 
Most studies of the determinants of success have based their conclusions upon comparisons of the exceptional with the mediocre.

The downside of commitment is that if you make what happen to be the wrong commitments, it can take a long time to undo them and make new ones.
The strategy paradox, then, arises from the collision of commitment and uncertainty.
Success is very often a result of having made what \textit{turned out to be} the right commitments, while failed strategies, which can be similar in many ways to successful ones, are based on what \textit{turned out to be} the wrong commitments.
In other words, the strategy paradox is a consequence of the need to commit to a strategy despite the deep uncertainty surrounding which strategy to commit to. Call this \textit{strategic} uncertainty.

Firms that avoid strategic risk survive but do not prosper. Firms that accept strategic risk reap either great reward or utter ruin. 

However, adaptability is far less useful than we might like. Specifically, adaptability is viable only when the pace of organizational change matches the pace of environmental change.
When the environment chaenges either faster or slower than the organization, adaptability is no longer sufficient. 
As a result, adaptability cannot expect to resolve or even mitigate the strategy paradox.

\end{document}